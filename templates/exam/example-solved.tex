%%% Uncomment only one of the next two lines (depending on whether you want the answers printed)
%%\documentclass[addpoints,12pt]{exam}
\documentclass[answers,addpoints,12pt]{exam}

\usepackage[usenames, dvipsnames]{color} % defines a new color
%\definecolor{SolutionColor}{rgb}{0.8,0.9,1} % light blue

\renewcommand{\solutiontitle}{\noindent\textbf{Answer: }}

%% How to print correct answer choices:
   \CorrectChoiceEmphasis{\itshape\bfseries} %% <-- bold italics
%%    
%% \CorrectChoiceEmphasis{\color{ForestGreen}\bfseries} %% <-- green bold

\pointsinmargin
\pointpoints{pt}{pts}
\marginpointname{pts}

\makeatletter
\newif\ifanswers
\@ifclasswith{exam}{answers}{\answerstrue}{\answersfalse}
\makeatother
\newcommand{\scratchpage}{%
  \ifanswers % do nothing
  \else \newpage \thispagestyle{empty} \begin{center} -- scratch -- \end{center} \fi}

\newcommand{\foo}{\ifanswers fooone\else footwo\fi}

%% Change geometry if you want:
%% \usepackage[top=2cm, left=2cm,right=2cm,bottom=1cm]{geometry}%

\usepackage{amsmath}
\usepackage{amsthm,amssymb}
\usepackage{mathtools}
\usepackage{url,multicol,enumerate}
\usepackage{tikz}

\theoremstyle{remark}
\newtheorem{theorem}{Theorem}
\newtheorem*{prop}{Proposition}
\newtheorem{problem}{Problem}
\newtheorem*{prob}{Problem}
\newtheorem*{answer}{{\bf Answer}}
\newtheorem*{answers}{{\bf Answers}}
\newtheorem*{explanation}{{\bf Explanation}}
\newtheorem*{hint}{{\it Hint}}
\newtheorem*{ex}{Exercise}


%% Some of my own personal favoriate macros... (remove these if you want)
\renewcommand{\vec}[1]{\mathbf{#1}}
%%       To make a boldface vector, use backslash v in front of the 
%%       letter and add a new command for that letter here or in 
%%       the macros.tex file:
\newcommand\rank{\ensuremath{\operatorname{rank}}}
\newcommand\nullity{\ensuremath{\operatorname{nullity}}}
\newcommand{\<}{\ensuremath{\langle}}
\renewcommand{\>}{\ensuremath{\rangle}}
\newcommand{\ur}{\ensuremath{\underline{\mathrm{r}}}}
\newcommand{\uT}{\ensuremath{\underline{\mathrm{T}}}}
\newcommand{\uF}{\ensuremath{\underline{\mathrm{F}}}}
\newcommand{\uN}{\ensuremath{\underline{\mathrm{N}}}}
\newcommand{\ui}{\ensuremath{\underline{\mathrm{i}}}}
\newcommand{\uj}{\ensuremath{\underline{\mathrm{j}}}}
\newcommand{\ua}{\ensuremath{\underline{\mathrm{a}}}}
\newcommand{\ub}{\ensuremath{\underline{\mathrm{b}}}}
\newcommand{\un}{\ensuremath{\underline{\mathrm{n}}}}
\newcommand{\uv}{\ensuremath{\underline{\mathrm{v}}}}
\newcommand{\R}{\ensuremath{\mathbb{R}}}
\newcommand\va{\vec{a}}
\newcommand\vb{\vec{b}}
\newcommand\vu{\vec{u}}
\newcommand\vv{\vec{v}}
\newcommand\vw{\vec{w}}
\newcommand\vs{\vec{s}}
\newcommand\vx{\vec{x}}
\newcommand\vy{\vec{y}}
\newcommand\vz{\vec{z}}
\newcommand\vzero{\vec{0}}

\newcommand{\dotsize}{1pt}
\newcommand{\Heq}{\ensuremath{ \; \stackrel{\mathrm{H}}{=}} \; }

\pagestyle{foot}
%%% Running footer will have a space for page score (if this is not the solution key)
\ifanswers  %% do nothing
\else
\runningfooter{}{}{Score for this page: \makebox[1in]{\hrulefill} out of \pointsonpage{\thepage}}
\fi


\begin{document}

\noindent {\bf MATH 317: SPRING 2016}\hfill \ifanswers {\bf ANSWERS} \hfill {\bf EXAM 1} 
\else {\bf EXAM 1}

    \begin{center}
      \fbox{\fbox{\parbox{7in}{\centering
            {\bf RULES}
            \begin{itemize}
            \item No books or notes or calculators allowed.
            \item No bathroom breaks until after you have completed and turned in your test.
            \item Out of consideration for your classmates, do not make
              disturbing noises during the exam.
            \item {\bf Phones and other electronic devices must be off or in silent mode.}
      \end{itemize}}}}
    \end{center}
    \vspace{0.2in}

    \vskip2cm

    \noindent    {\it Cheating will not be tolerated.}  If there is any indication that a
    student may have given or received unauthorized aid on this test, the case 
    will be handed over to the ISU Office of Judicial Affairs.
    When you finish the exam, please sign the following statement acknowledging that you understand 
    this policy:\\
    \\
    ``On my honor as a student I,
    \underline{\phantom{XXXXXXXXXXXXXXXXXXXXXX}}, have neither
    given nor received unauthorized aid on this exam.''
    \hbox{} \hskip .25cm {\small (print name clearly)}\\
    \\
    \vspace{0.1in}
    \makebox[\textwidth]{Signature:\enspace\hrulefill ~Date: \underline{\phantom{x} 2016-02-19 \phantom{x}}}


  \newpage
~
\vskip4cm
\thispagestyle{empty}
{\large \begin{center} \gradetable[v][pages] \end{center}  }
~
  \newpage
      \fi

\begin{questions} % Begins the questions environment

  %%%% Problem 1.
  \question
  Let $\vx = (x_1, x_2, \dots, x_n)$ and $\vy = (y_1, y_2, \dots, y_n)$ be vectors in $\R^n$.
  \begin{parts}
    \part[3] Define the dot product of $\vx$ and $\vy$.
    \begin{solution}
      \[\vx \cdot \vy = x_1y_1 + x_2 y_2 + \cdots + x_n y_n.\]
    \end{solution}
    \vspace{\stretch{1}}

    \part[3] What does it mean to say that $\vx$ and $\vy$ are \emph{parallel}?

    \begin{solution} The vectors $\vx$ and $\vy$ are \emph{parallel} if they are scalar multiples of
      each other. That is, there exists a real number $c$ such that $\vx = c \,\vy$.
    \end{solution}
    \vspace{\stretch{1}}

    \part[4]
    What does it mean to say that $\vx$ and $\vy$ are \emph{orthogonal}?
    \begin{solution} The vectors $\vx$ and $\vy$ are \emph{orthogonal} if their dot product is 0.
      That is, $\vx \cdot \vy=0$.
    \end{solution}

    \vspace{\stretch{1}}

    \part[4]
    Suppose $\vv_1, \vv_2, \dots, \vv_k$ are also vectors in $\R^n$.
    What does it mean to say that $\vx$ is a \emph{linear combination} of the vectors 
    $\vv_1, \vv_2, \dots, \vv_k$?
    \begin{solution}
      $\vx$ is a \emph{linear combination} of the vectors 
    $\vv_1, \vv_2, \dots, \vv_k$ if there exist scalars $c_1, c_2, \dots, c_n$ such that
      \[
      \vx = c_1 \vv_1 +c_2 \vv_2 + \cdots +c_n \vv_n.
      \]
    \end{solution}
    \vspace{\stretch{1}}

  \end{parts}

\newpage


%%%% Problem 2. %%%%%%%%%%%%%%%%%%%%%%%%%%%%%%%%%%%%%%%%%%%%%
\question[10]
Recall Prop.~2.1 of our text says that the dot product satisfies the following properties:
for all $\vx, \vy, \vz \in \R^n$ and $c \in \R$,
\begin{enumerate}
\item $\vx \cdot \vy = \vy \cdot \vx$;
\item $\vx \cdot \vx = \|\vx\|^2 \geq 0$, with equality if and only if $\vx = 0$;
\item $(c\vx) \cdot \vy = c(\vx \cdot \vy)$;
\item $\vx \cdot (\vy + \vz) = \vx \cdot \vy + \vx \cdot \vz$.
\end{enumerate}

Prove that if $\vx$ is orthogonal to each of the vectors 
$\vv_1, \vv_2, \dots, \vv_k$, then  $\vx$ is orthogonal to every linear
combination of the vectors $\vv_1,\dots, \vv_k$.  (For full credit,
identify places in your proof where properties from the list above are used;
to refer to these properties, use the letters given above.)

\begin{solution}
  Suppose $\vx$ is orthogonal to each of the vectors $\vv_1,\dots, \vv_k$. Then,
  \begin{alignat*}{2}
    \vx \cdot (c_1 \vv_1 + \cdots + c_k \vv_k) 
    &= \vx\cdot (c_1 \vv_1) + \cdots + \vx\cdot (c_k \vv_k)
    &\quad & (\because \text{Prop.~2.1 (4)})\\
    &= c_1 (\vx\cdot \vv_1) + \cdots + c_k (\vx\cdot \vv_k)
    &\quad & (\because \text{Prop.~2.1 (1) and (3)})\\
    &= c_1 0 + \cdots + c_k 0 
    &\quad & (\because \vx\cdot \vv_i = 0 \text{ for all $1\leq i\leq k$})\\
    &= 0.
  \end{alignat*}
\end{solution}

\vspace{\stretch{1}}


%%%% Problem 3. %%%%%%%%%%%%%%%%%%%%%%%%%%%%%%%%%%%%%%%%%%%%%

\question[5] 
\label{qu:3}
State a theorem about existence and uniqueness of solutions to the system $A
    \vx = \vb$. You may state more than one theorem if you wish, but
    quality is better than quantity. Only write what you know is true and 
    {\bf carefully state your assumptions}.  If you make a broad statement that, in fact, only applies
    under a narrow set of conditions, and you leave out those conditions, then you will not receive
    very much credit. (Use only the space provided below.)
    
\begin{solution}
There is more than one good answer to this.  Here's one possibility, which is probably the most
helpful when answering the questions on the next page.  
(Other possible answers are given on the last page below.)

Let $A\in \R^{m\times n}$. Then the equation $A\vx = \vb$ 
\begin{enumerate}
\item is consistent for all $\vb \in \R^m$ when $\rank(A)=m$ (=number of rows of $A$);
\item has at most one solution when $\rank(A)=n$ (=number of columns of $A$).
\end{enumerate}
Recall that $\rank(A)=m$ iff the echelon form of $A$ has 
no rows consisting of all zeros; 
$\rank(A)=n$ iff the echelon form of $A$ has 
a pivot in every column.

\end{solution}

\vspace{\stretch{1}}

\newpage


%%%% Problem 4. %%%%%%%%%%%%%%%%%%%%%%%%%%%%%%%%%%%%%%%%%%%%%%%
\question For the given matrix, circle the letters corresponding to true statements in each case.
Select from the following statements:
\begin{enumerate}[A.]
        \item For every $\vb \in \R^m$, $A\vx = \vb$ is consistent.
        \item For every $\vb \in \R^m$, $A\vx = \vb$ is inconsistent.
        \item For every $\vb \in \R^m$, $A\vx = \vb$ has exactly one solution.
        \item For every $\vb \in \R^m$, $A\vx = \vb$ has infinitely many solutions.
        \item There exists $\vb \in \R^m$ such that $A\vx = \vb$ is consistent.
        \item There exists $\vb \in \R^m$ such that $A\vx = \vb$ is inconsistent.
        \item There exists $\vb \in \R^m$ such that $A\vx = \vb$ has exactly one solution.
        \item There exists $\vb \in \R^m$ such that $A\vx = \vb$ has infinitely many solutions.
\end{enumerate}
\begin{parts}
  \part[4] If the matrix $A$ has reduced echelon form 
  $\begin{bmatrix*}[r]       	
      1 & 0 & 0 \\
      0 & 1 & 0 \\
      0 & 0 & 1
  \end{bmatrix*}$, \\\\then
      which of the statements (a)--(h) above is true? (Select all that apply.)\\\\
      \begin{oneparchoices} 
        \CorrectChoice % A
        \choice % B 
        \CorrectChoice % C
        \choice % D
        \CorrectChoice % E
        \choice % F
        \CorrectChoice % G
        \choice % H
      \end{oneparchoices}

      \vspace{\stretch{1}}       

      \part[4] If the matrix $A$ has reduced echelon form 
      $\begin{bmatrix*}[r]       	
        1 & 0 & 0 \\
        0 & 1 & 0 \\
        0 & 0 & 1 \\
        0 & 0 & 0
      \end{bmatrix*}$, \\\\then
      which of the statements (a)--(h) above is true? (Select all that apply.)\\\\
      \begin{oneparchoices}
        \choice % A
        \choice % B 
        \choice % C
        \choice % D
        \CorrectChoice % E
        \CorrectChoice % F
        \CorrectChoice % G
        \choice % H
      \end{oneparchoices}

      \vspace{\stretch{1}}       

      \part[4] If the matrix $A$ has reduced echelon form 
      $\begin{bmatrix*}[r]1 & 0 & 1 \\0 & 1 & 2 \end{bmatrix*}$, \\\\then
       which of the statements (a)--(h) above is true? (Select all that apply.)\\\\
      \begin{oneparchoices} 
        \CorrectChoice % A
        \choice % B 
        \choice % C
        \CorrectChoice % D
        \CorrectChoice % E
        \choice % F
        \choice % G
        \CorrectChoice % H
      \end{oneparchoices}

      \vspace{\stretch{1}}       

      \part[4] If the matrix %$A\in \R^{4\times 6}$
      $A$ has reduced echelon form 
      $\begin{bmatrix*}[r]       	
      1 & 0& -1 & 2 & 0 & 1\\
      0 & 1 & 1 & 3 & 0& -2\\
      0 & 0 & 0 & 0 & 1& -1\\
      0 & 0 & 0 & 0 & 0 & 0\end{bmatrix*}$, \\\\then
       which of the statements (a)--(h) above is true? (Select all that apply.)\\\\
      \begin{oneparchoices} 
        \choice % A
        \choice % B 
        \choice % C
        \choice % D
        \CorrectChoice % E
        \CorrectChoice % F
        \choice % G
        \CorrectChoice % H
      \end{oneparchoices}

    \end{parts}

\newpage
    %%%% Problem 5.
\question
    Let $A \in \R^{m\times n}$, $B \in \R^{n\times m}$, $C \in \R^{n\times m}$, and $\vb\in \R^m$. 
    For each statement below, either prove the claim or write FALSE and give a counter-example. 
    \begin{parts}
      
   \part[5]
    {\bf Claim:} If $AB = I_m$, then a solution to $A\vx = \vb$, if it exists, is unique.

      \begin{solution} FALSE.  
        Let $A = \begin{bmatrix*}[r] 1 &1 \end{bmatrix*}$, and suppose $\vb = [0]$.
        Then there are infinitely many solutions to $A\vx =\vb$, since every
        $\vx = (x_1, x_2)$ satisfying $x_1 + x_2 = 0$ is a solution. 
      \end{solution}

      \vspace{\stretch{1}}       

   \part[5]
    {\bf Claim:} If $CA = I_n$, then a solution to $A\vx = \vb$, if it exists, is unique.

      \begin{solution}
        Suppose $A\vx = \vb$ has a solution, call it $\vx'$.
        Suppose $\vx''$ is another solution, so that $A\vx'' = \vb$. Then 
        $A\vx' = A\vx''$, so
        $BA\vx' = BA\vx''$, so
        $I_n\vx' = I_n\vx''$, so
        $\vx' = \vx''$.
      \end{solution}

      \vspace{\stretch{1}}       

    \part[5]
    {\bf Claim:} If $A\vx = \vzero$ has only the trivial solution $\vx = \vzero$, then 
    for every $\vb\in \R^m$ there is exactly one solution to $A\vx = \vb$.

      \begin{solution}
        FALSE. 
        Let $A = \begin{bmatrix*}[r] 1\\0 \end{bmatrix*}$.
        Then the only solution to $A\vx =\vzero$ is $\vx =[0]$. That is,
        \[\begin{bmatrix*}[r] 1\\0 \end{bmatrix*} 
        \begin{bmatrix*}[r] x_1 \end{bmatrix*} 
        =\begin{bmatrix*}[r] 0\\0 \end{bmatrix*}
        \quad \text{ if and only if } \quad x_1 = 0.\]
        In this case, \emph{if the equation $A\vx = \vb$ has a solution}, then it is unique.
        However, it is not true that $A\vx = \vb$ always has a unique solution,
        since it may be inconsistent.
        For example, continuing with the matrix $A = \begin{bmatrix*}[r] 1\\0 \end{bmatrix*}$ above, 
        if $\vb = \begin{bmatrix*}[r] 1\\1\end{bmatrix*}$, then $A\vx = \vb$ has no solution.
      \end{solution}

      \vspace{\stretch{1}}       
 
     \end{parts}

  \newpage

  %%%% Problem 6.
  \question  Let 
  $A = \begin{bmatrix*}[r] 1 & 1 & 0\\ 0 & 1& -1\\ 1 & 1 & 1\end{bmatrix*}$
    and $\vb = \begin{bmatrix*}[r] 1\\2\\3\end{bmatrix*}$.
      \begin{parts}
        \part[8] Find $A^{-1}$.

        \begin{solution}
          \[A^{-1} = \begin{bmatrix*}[r] 2& -1& -1\\-1 & 1 & 1\\-1 & 0 &1\end{bmatrix*}\]
        \end{solution}

        \vspace{\stretch{1}}       

        \part[5] Use your answer to (a) to solve $A\vx = \vb$.

        \begin{solution}
          \[\vx = \begin{bmatrix*}[r]-3\\ 4\\ 2\end{bmatrix*} \]
        \end{solution}

        \vspace{\stretch{1}}       

        \part[2] %Let $\va_i$ denote the $i$-th column of the matrix $A$.  
        Use your answer to (b) to express $\vb$ as a linear combination of the columns of $A$.\\
        (Fill in the blanks with your answers.)\\\\

        $\begin{bmatrix*}[r] 1\\2\\3\end{bmatrix*} =$
          \fillin[$-3$][1cm] $\begin{bmatrix*}[r] 1\\0\\1\end{bmatrix*} +$
            \fillin[$ 4$][1cm] $\begin{bmatrix*}[r] 1\\1\\1\end{bmatrix*} +$
              \fillin[$ 2$][1cm] $\begin{bmatrix*}[r] 0\\-1\\1\end{bmatrix*}$ 
                
      \end{parts}

\end{questions}

\ifanswers
\newpage 
Alternative answers to Question \ref{qu:3}:

\begin{itemize}
\item 
Suppose the system $A\vx = \vb$ is consistent. Then it has a unique solution if
and only if the associated homogeneous system $A\vx = \vzero$ has only the trivial solution. This
happens exactly when $\rank(A)$ is the number of columns of $A$.

\item Let $A\in \R^{n\times n}$. The following are equivalent:
\begin{enumerate}
\item $A$ is nonsingular.
\item $A\vx = \vzero$ has only the trivial solution.
\item For every $\vb \in \R^n$, the equation $A\vx = \vb$ has a solution (indeed, a unique solution).
\end{enumerate}
\end{itemize}

\else % do nothing
\fi

\scratchpage

\scratchpage


\end{document}


%%% inserting a grid for graphing
%% \smallskip\fillwithgrid{1in}
